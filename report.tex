\documentclass{article}

\usepackage[T1]{fontenc}
%\usepackage[norwegian]{babel}
\usepackage{graphicx}
\usepackage{amsmath}
\usepackage{amsfonts}
\usepackage{color}
\usepackage{listings}
\usepackage{algorithm}
\usepackage{algorithmic}
\usepackage{url}

\newcommand{\differential}[2]{\frac{\mathrm{d}#1}{\mathrm{d}#2}}
\newcommand{\qqq}{\qquad\qquad\qquad}
\newcommand{\f}[2]{\frac{#1}{#2}}
\newcommand{\bs}[1]{$\boldsymbol #1$}
\newcommand{\bsa}[1]{\boldsymbol #1}

\lstset{keywordstyle=\color{Red},%\bfseries,
    basicstyle=\small\ttfamily,
    %identifierstyle=\color{NavyBlue},
    commentstyle=\color{Red}\ttfamily,
    stringstyle=\rmfamily,
    %numbers=none,%left,%
    %numberstyle=\scriptsize,%\tiny
    %stepnumber=5,
    numbersep=8pt,
    %showstringspaces=false,
    breaklines=true,
    %frameround=ftff,
    frame=single
    frame=L
}

\begin{document}

\author{Andreas V�vang Solbr�\\Gunnar Raaen\\Anders Hafreager}
\title{INF5620 - a numerical analysis of the two dimensional wave equation}
\maketitle
\newpage

\section*{Abstract}

\section*{Theory}
\begin{align}
\frac{\partial^2 u}{\partial t^2} + b\frac{\partial u}{\partial t} =
\frac{\partial}{\partial x}\left( q (x,y)
\frac{\partial u}{\partial x}\right) +
\frac{\partial}{\partial y}\left( q (x,y)
\frac{\partial u}{\partial y}\right) + f(x,y,t),
\end{align}

\section*{Numerical implementation}

\section*{Verification}
\subsection*{Constant solution}
\subsection*{Exact 1D solution}
\subsection*{Standing wave}
\subsection*{Manufactured solution}
\section*{Applications}
\subsection*{Tsunami}
In addition to what HPL says (ex 14), check out project 4 from FYS3150 in 2008:\\
\url{http://bit.ly/3150tsunami}
\subsection*{Electromagnetic wave}
\subsection*{Double slit}

\section*{Visualization}


\end{document}